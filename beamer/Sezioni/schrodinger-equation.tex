\section{Time-Dependent Schrödinger Equation}

\begin{frame}{Approach to the Schrödinger equation}
    Moving in the quantum realm, our next goal is to approximate the solution of a \textbf{1D time-dependent \textcolor{BrickRed}{Schrödinger equation}}:

    \begin{equation*}
        \fcolorbox{BrickRed}{white}{\text{$\displaystyle i\hbar\frac{\partial\psi}{\partial t}=-\frac{\hbar^2}{2m}\frac{\partial^2\psi}{\partial x^2}+V(x)\psi(x,t)$}}
    \end{equation*}
\end{frame}

\begin{frame}{New features}
    Three main differences from what we saw before:

    \vfill

    \begin{enumerate}
        \item It appears a \underline{linear term}.
        
        We must introduce a \textbf{weighted mass matrix}:
        
        \begin{equation*}
            V:V_{i,j}=\int_\Omega V(x)\phi_j(x)\phi_i(x)dx
        \end{equation*}

        \vspace{0.25cm}
        
        \item The PDE is of \underline{first order} in time.
        
        The best method to exploit is the \textbf{Crank-Nicholson} one:
        
        \begin{equation*}
            \left[M+\frac{i\Delta t}{2\hbar}\left(\frac{\hbar^2}{2m}A+V\right)\right]\psi^{(n+1)}=\left[M-\frac{i\Delta t}{2\hbar}\left(\frac{\hbar^2}{2m}A+V\right)\right]\psi^{(n)}
        \end{equation*}

        \vspace{0.25cm}

        \item The presence of the imaginary unit introduces \underline{complex-valued wavefunctions}.
    \end{enumerate}
\end{frame}

\begin{frame}{How to handle imaginary part}
    
\end{frame}